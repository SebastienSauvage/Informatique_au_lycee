\titre{Requêtes en langage SQL}{Correction}

\exercice{Exercice p2}{
\begin{enumerate}
    \item 
    \fbox{
    SELECT DISTINCT categorie FROM evolution ORDER BY categorie ASC;
    }
    \item 
    \fbox{
    SELECT COUNT (DISTINCT categorie) AS NbCategories FROM evolution;
    }
    \item 
    \fbox{
    SELECT code FROM ville WHERE nom = "Caullery";
    }
    \item 
    \fbox{
    SELECT * FROM evolution WHERE code = "59140" ORDER BY effectif ASC;
    }
    \item 
    \fbox{
    SELECT code FROM evolution WHERE effectif > 2000;
    }
    \item 
    \fbox{
    \begin{minipage}{1\linewidth}
    SELECT SUM (categorie) AS NbAgricultrices FROM evolution WHERE categorie = "Agriculteurs Exploitants" AND genre = "Femmes";
    \end{minipage}
    }
    \item 
    \fbox{
    SELECT AVG(effectif) AS MoyenneEmployes FROM evolution WHERE categorie = "Employés";
    }
    
\end{enumerate}
}
\exercice{Exercice p3}{
\begin{enumerate}
    \item 
    \fbox{
    \begin{minipage}{1\linewidth}
    SELECT evolution.categorie, evolution.genre, evolution.effectif FROM evolution JOIN ville ON evolution.code = ville.code WHERE ville.nom = "Caullery";
    \end{minipage}
    }
    \item 
    \fbox{
    \begin{minipage}{1\linewidth}
    SELECT ville.nom, evolution.categorie, evolution.genre FROM ville JOIN evolution ON evolution.code = ville.code WHERE evolution.effectif = 0 ORDER BY evolution.categorie ASC;
    \end{minipage}
    }
    \item 
    \fbox{
    \begin{minipage}{1\linewidth}
    SELECT ville.nom FROM ville JOIN evolution ON ville.code = evolution.code WHERE evolution.effectif > 2000;
    \end{minipage}
    }
    \item 
    \fbox{
    \begin{minipage}{1\linewidth}
    SELECT ville.nom FROM ville JOIN evolution ON ville.code = evolution.code WHERE evolution.effectif > 2000 ORDER BY evolution.effectif ASC;
    \end{minipage}
    }
    \item 
    \fbox{
    \begin{minipage}{1\linewidth}
    SELECT ville.nom FROM ville JOIN evolution ON ville.code = evolution.code WHERE evolution.effectif > 2000 AND ville.nom <> "Lille" ORDER BY evolution.effectif ASC;
    \end{minipage}
    }
\end{enumerate}
}

\exercice{Exercice p4}{
\begin{enumerate}
    \item 
    \fbox{
    \begin{minipage}{1\linewidth}

    \end{minipage}
    }
    
    \item 
    \fbox{
    \begin{minipage}{1\linewidth}

    \end{minipage}
    }
    
    \item 
    \fbox{
    \begin{minipage}{1\linewidth}

    \end{minipage}
    }
    
    \item 
    \fbox{
    \begin{minipage}{1\linewidth}

    \end{minipage}
    }
    
    \item 
    \fbox{
    \begin{minipage}{1\linewidth}

    \end{minipage}
    }
    
    \item 
    \fbox{
    \begin{minipage}{1\linewidth}

    \end{minipage}
    }
    
    \item 
    \fbox{
    \begin{minipage}{1\linewidth}

    \end{minipage}
    }
    
\end{enumerate}
}
